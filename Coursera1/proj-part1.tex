% Options for packages loaded elsewhere
\PassOptionsToPackage{unicode}{hyperref}
\PassOptionsToPackage{hyphens}{url}
%
\documentclass[
]{article}
\usepackage{amsmath,amssymb}
\usepackage{lmodern}
\usepackage{iftex}
\ifPDFTeX
  \usepackage[T1]{fontenc}
  \usepackage[utf8]{inputenc}
  \usepackage{textcomp} % provide euro and other symbols
\else % if luatex or xetex
  \usepackage{unicode-math}
  \defaultfontfeatures{Scale=MatchLowercase}
  \defaultfontfeatures[\rmfamily]{Ligatures=TeX,Scale=1}
\fi
% Use upquote if available, for straight quotes in verbatim environments
\IfFileExists{upquote.sty}{\usepackage{upquote}}{}
\IfFileExists{microtype.sty}{% use microtype if available
  \usepackage[]{microtype}
  \UseMicrotypeSet[protrusion]{basicmath} % disable protrusion for tt fonts
}{}
\makeatletter
\@ifundefined{KOMAClassName}{% if non-KOMA class
  \IfFileExists{parskip.sty}{%
    \usepackage{parskip}
  }{% else
    \setlength{\parindent}{0pt}
    \setlength{\parskip}{6pt plus 2pt minus 1pt}}
}{% if KOMA class
  \KOMAoptions{parskip=half}}
\makeatother
\usepackage{xcolor}
\usepackage[margin=1in]{geometry}
\usepackage{color}
\usepackage{fancyvrb}
\newcommand{\VerbBar}{|}
\newcommand{\VERB}{\Verb[commandchars=\\\{\}]}
\DefineVerbatimEnvironment{Highlighting}{Verbatim}{commandchars=\\\{\}}
% Add ',fontsize=\small' for more characters per line
\usepackage{framed}
\definecolor{shadecolor}{RGB}{248,248,248}
\newenvironment{Shaded}{\begin{snugshade}}{\end{snugshade}}
\newcommand{\AlertTok}[1]{\textcolor[rgb]{0.94,0.16,0.16}{#1}}
\newcommand{\AnnotationTok}[1]{\textcolor[rgb]{0.56,0.35,0.01}{\textbf{\textit{#1}}}}
\newcommand{\AttributeTok}[1]{\textcolor[rgb]{0.77,0.63,0.00}{#1}}
\newcommand{\BaseNTok}[1]{\textcolor[rgb]{0.00,0.00,0.81}{#1}}
\newcommand{\BuiltInTok}[1]{#1}
\newcommand{\CharTok}[1]{\textcolor[rgb]{0.31,0.60,0.02}{#1}}
\newcommand{\CommentTok}[1]{\textcolor[rgb]{0.56,0.35,0.01}{\textit{#1}}}
\newcommand{\CommentVarTok}[1]{\textcolor[rgb]{0.56,0.35,0.01}{\textbf{\textit{#1}}}}
\newcommand{\ConstantTok}[1]{\textcolor[rgb]{0.00,0.00,0.00}{#1}}
\newcommand{\ControlFlowTok}[1]{\textcolor[rgb]{0.13,0.29,0.53}{\textbf{#1}}}
\newcommand{\DataTypeTok}[1]{\textcolor[rgb]{0.13,0.29,0.53}{#1}}
\newcommand{\DecValTok}[1]{\textcolor[rgb]{0.00,0.00,0.81}{#1}}
\newcommand{\DocumentationTok}[1]{\textcolor[rgb]{0.56,0.35,0.01}{\textbf{\textit{#1}}}}
\newcommand{\ErrorTok}[1]{\textcolor[rgb]{0.64,0.00,0.00}{\textbf{#1}}}
\newcommand{\ExtensionTok}[1]{#1}
\newcommand{\FloatTok}[1]{\textcolor[rgb]{0.00,0.00,0.81}{#1}}
\newcommand{\FunctionTok}[1]{\textcolor[rgb]{0.00,0.00,0.00}{#1}}
\newcommand{\ImportTok}[1]{#1}
\newcommand{\InformationTok}[1]{\textcolor[rgb]{0.56,0.35,0.01}{\textbf{\textit{#1}}}}
\newcommand{\KeywordTok}[1]{\textcolor[rgb]{0.13,0.29,0.53}{\textbf{#1}}}
\newcommand{\NormalTok}[1]{#1}
\newcommand{\OperatorTok}[1]{\textcolor[rgb]{0.81,0.36,0.00}{\textbf{#1}}}
\newcommand{\OtherTok}[1]{\textcolor[rgb]{0.56,0.35,0.01}{#1}}
\newcommand{\PreprocessorTok}[1]{\textcolor[rgb]{0.56,0.35,0.01}{\textit{#1}}}
\newcommand{\RegionMarkerTok}[1]{#1}
\newcommand{\SpecialCharTok}[1]{\textcolor[rgb]{0.00,0.00,0.00}{#1}}
\newcommand{\SpecialStringTok}[1]{\textcolor[rgb]{0.31,0.60,0.02}{#1}}
\newcommand{\StringTok}[1]{\textcolor[rgb]{0.31,0.60,0.02}{#1}}
\newcommand{\VariableTok}[1]{\textcolor[rgb]{0.00,0.00,0.00}{#1}}
\newcommand{\VerbatimStringTok}[1]{\textcolor[rgb]{0.31,0.60,0.02}{#1}}
\newcommand{\WarningTok}[1]{\textcolor[rgb]{0.56,0.35,0.01}{\textbf{\textit{#1}}}}
\usepackage{graphicx}
\makeatletter
\def\maxwidth{\ifdim\Gin@nat@width>\linewidth\linewidth\else\Gin@nat@width\fi}
\def\maxheight{\ifdim\Gin@nat@height>\textheight\textheight\else\Gin@nat@height\fi}
\makeatother
% Scale images if necessary, so that they will not overflow the page
% margins by default, and it is still possible to overwrite the defaults
% using explicit options in \includegraphics[width, height, ...]{}
\setkeys{Gin}{width=\maxwidth,height=\maxheight,keepaspectratio}
% Set default figure placement to htbp
\makeatletter
\def\fps@figure{htbp}
\makeatother
\setlength{\emergencystretch}{3em} % prevent overfull lines
\providecommand{\tightlist}{%
  \setlength{\itemsep}{0pt}\setlength{\parskip}{0pt}}
\setcounter{secnumdepth}{-\maxdimen} % remove section numbering
\ifLuaTeX
  \usepackage{selnolig}  % disable illegal ligatures
\fi
\IfFileExists{bookmark.sty}{\usepackage{bookmark}}{\usepackage{hyperref}}
\IfFileExists{xurl.sty}{\usepackage{xurl}}{} % add URL line breaks if available
\urlstyle{same} % disable monospaced font for URLs
\hypersetup{
  pdftitle={Statistical Interference Сourse Project. Part 1},
  pdfauthor={Daria Kravets},
  hidelinks,
  pdfcreator={LaTeX via pandoc}}

\title{Statistical Interference Сourse Project. Part 1}
\author{Daria Kravets}
\date{}

\begin{document}
\maketitle

\hypertarget{overview}{%
\subsection{Overview}\label{overview}}

This project involves investigating the properties of the exponential
distribution in R and comparing it with the Central Limit Theorem. The
exponential distribution can be simulated in R using the rexp(n, lambda)
function, where lambda is the rate parameter. For this project, lambda
will be set at 0.2 for all simulations.

The aim is to investigate the distribution of averages of 40
exponentials by conducting a thousand simulations. Specifically, the
project demonstrates a difference between samle characteristics, such as
mean and variance, and theoretical values of it.

Finally, the normality of the distribution will be tested using
appropriate methods such as a histogram or QQ plot to demonstrate that
the distribution is approximately normal.

\hypertarget{sample-simulation}{%
\subsection{Sample simulation}\label{sample-simulation}}

First, for having a stable random sample we will set the seed:

\begin{Shaded}
\begin{Highlighting}[]
\FunctionTok{set.seed}\NormalTok{(}\DecValTok{5327}\NormalTok{)}
\end{Highlighting}
\end{Shaded}

Then we need to set some values needed for our project

\begin{Shaded}
\begin{Highlighting}[]
\NormalTok{lambda }\OtherTok{\textless{}{-}} \FloatTok{0.2}
\NormalTok{n }\OtherTok{\textless{}{-}} \DecValTok{40}
\NormalTok{size }\OtherTok{\textless{}{-}} \DecValTok{1000}
\end{Highlighting}
\end{Shaded}

And finally we can set our sample by producing random values:

\begin{Shaded}
\begin{Highlighting}[]
\NormalTok{simulated\_sample }\OtherTok{=} \ConstantTok{NULL}
\ControlFlowTok{for}\NormalTok{ (i }\ControlFlowTok{in} \DecValTok{1} \SpecialCharTok{:}\NormalTok{ size) simulated\_sample }\OtherTok{=} \FunctionTok{c}\NormalTok{(simulated\_sample, }\FunctionTok{mean}\NormalTok{(}\FunctionTok{rexp}\NormalTok{(n, lambda)))}
\end{Highlighting}
\end{Shaded}

\hypertarget{sample-mean-and-theoretical-mean-comparisson}{%
\subsection{Sample mean and Theoretical mean
comparisson}\label{sample-mean-and-theoretical-mean-comparisson}}

Calculating mean from the simulations

\begin{Shaded}
\begin{Highlighting}[]
\NormalTok{sample\_mean }\OtherTok{\textless{}{-}} \FunctionTok{mean}\NormalTok{(simulated\_sample)}
\NormalTok{sample\_mean}
\end{Highlighting}
\end{Shaded}

\begin{verbatim}
## [1] 4.97759
\end{verbatim}

Calculating theoretical mean using formula

\begin{Shaded}
\begin{Highlighting}[]
\NormalTok{theor\_mean }\OtherTok{\textless{}{-}}\NormalTok{ lambda}\SpecialCharTok{\^{}}\NormalTok{(}\SpecialCharTok{{-}}\DecValTok{1}\NormalTok{)}
\NormalTok{theor\_mean}
\end{Highlighting}
\end{Shaded}

\begin{verbatim}
## [1] 5
\end{verbatim}

Comparing of this two values by calculating their difference

\begin{Shaded}
\begin{Highlighting}[]
\NormalTok{mean\_dif }\OtherTok{\textless{}{-}} \FunctionTok{abs}\NormalTok{(sample\_mean }\SpecialCharTok{{-}}\NormalTok{ theor\_mean)}
\NormalTok{mean\_dif}
\end{Highlighting}
\end{Shaded}

\begin{verbatim}
## [1] 0.02241024
\end{verbatim}

From this calculations, it is clear that our result has only slight
difference from a theoretical value.

\hypertarget{sample-variance-and-theoretical-variance-comparisson}{%
\subsection{Sample variance and Theoretical variance
comparisson}\label{sample-variance-and-theoretical-variance-comparisson}}

Calculating variance from the simulations

\begin{Shaded}
\begin{Highlighting}[]
\NormalTok{sample\_var }\OtherTok{\textless{}{-}}\FunctionTok{var}\NormalTok{(simulated\_sample)}
\NormalTok{sample\_var}
\end{Highlighting}
\end{Shaded}

\begin{verbatim}
## [1] 0.6452081
\end{verbatim}

Calculating theoretical variance using formula

\begin{Shaded}
\begin{Highlighting}[]
\NormalTok{theor\_var }\OtherTok{\textless{}{-}} \DecValTok{1} \SpecialCharTok{/}\NormalTok{ (lambda}\SpecialCharTok{\^{}}\DecValTok{2} \SpecialCharTok{*}\NormalTok{ n)}
\NormalTok{theor\_var}
\end{Highlighting}
\end{Shaded}

\begin{verbatim}
## [1] 0.625
\end{verbatim}

Comparing of this two values by calculating their difference

\begin{Shaded}
\begin{Highlighting}[]
\NormalTok{var\_dif }\OtherTok{\textless{}{-}} \FunctionTok{abs}\NormalTok{(sample\_var }\SpecialCharTok{{-}}\NormalTok{ theor\_var)}
\NormalTok{var\_dif}
\end{Highlighting}
\end{Shaded}

\begin{verbatim}
## [1] 0.02020813
\end{verbatim}

From this calculations, it is clear again that our result has only
slight difference from a theoretical value.

\hypertarget{distribution}{%
\subsection{Distribution}\label{distribution}}

By plotting a histogram and a normailty Q-Q plot, we can conclude
whether the distribution is normal.

\begin{Shaded}
\begin{Highlighting}[]
\NormalTok{xx }\OtherTok{\textless{}{-}} \FunctionTok{seq}\NormalTok{(}\FunctionTok{min}\NormalTok{(simulated\_sample), }\FunctionTok{max}\NormalTok{(simulated\_sample), }\AttributeTok{length=}\DecValTok{1000}\NormalTok{)}
\FunctionTok{hist}\NormalTok{(simulated\_sample, }\AttributeTok{breaks =}\NormalTok{ n, }\AttributeTok{freq=}\NormalTok{F, }\AttributeTok{col=}\StringTok{\textquotesingle{}\#0665a5\textquotesingle{}}\NormalTok{, }\AttributeTok{xlab=}\StringTok{\textquotesingle{}Means\textquotesingle{}}\NormalTok{, }\AttributeTok{ylab=}\StringTok{\textquotesingle{}Density\textquotesingle{}}\NormalTok{, }\AttributeTok{main=}\StringTok{\textquotesingle{}Histogram of means\textquotesingle{}}\NormalTok{)}
\FunctionTok{lines}\NormalTok{(xx, }\FunctionTok{dnorm}\NormalTok{(xx, }\AttributeTok{mean=}\NormalTok{sample\_mean, }\AttributeTok{sd=}\FunctionTok{sd}\NormalTok{(simulated\_sample)), }\AttributeTok{col=}\StringTok{\textquotesingle{}red\textquotesingle{}}\NormalTok{, }\AttributeTok{lwd=}\DecValTok{2}\NormalTok{)}
\end{Highlighting}
\end{Shaded}

\includegraphics{proj-part1_files/figure-latex/unnamed-chunk-10-1.pdf}

\begin{Shaded}
\begin{Highlighting}[]
\FunctionTok{qqnorm}\NormalTok{(simulated\_sample, }\AttributeTok{xlab=}\StringTok{\textquotesingle{}Theoretical quantiles\textquotesingle{}}\NormalTok{, }\AttributeTok{ylab=}\StringTok{\textquotesingle{}Sample quantiles\textquotesingle{}}\NormalTok{, }\AttributeTok{main =} \StringTok{\textquotesingle{}Q{-}Q plot\textquotesingle{}}\NormalTok{)}
\FunctionTok{qqline}\NormalTok{(simulated\_sample, }\AttributeTok{col=}\StringTok{\textquotesingle{}red\textquotesingle{}}\NormalTok{, }\AttributeTok{lwd=}\DecValTok{2}\NormalTok{)}
\end{Highlighting}
\end{Shaded}

\includegraphics{proj-part1_files/figure-latex/unnamed-chunk-11-1.pdf}

\textbf{Both plots are showing that the distribution is approximately
normal.}

\end{document}
